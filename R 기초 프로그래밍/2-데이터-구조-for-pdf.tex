% Options for packages loaded elsewhere
\PassOptionsToPackage{unicode}{hyperref}
\PassOptionsToPackage{hyphens}{url}
%
\documentclass[
]{article}
\usepackage{amsmath,amssymb}
\usepackage{lmodern}
\usepackage{iftex}
\ifPDFTeX
  \usepackage[T1]{fontenc}
  \usepackage[utf8]{inputenc}
  \usepackage{textcomp} % provide euro and other symbols
\else % if luatex or xetex
  \usepackage{unicode-math}
  \defaultfontfeatures{Scale=MatchLowercase}
  \defaultfontfeatures[\rmfamily]{Ligatures=TeX,Scale=1}
\fi
% Use upquote if available, for straight quotes in verbatim environments
\IfFileExists{upquote.sty}{\usepackage{upquote}}{}
\IfFileExists{microtype.sty}{% use microtype if available
  \usepackage[]{microtype}
  \UseMicrotypeSet[protrusion]{basicmath} % disable protrusion for tt fonts
}{}
\makeatletter
\@ifundefined{KOMAClassName}{% if non-KOMA class
  \IfFileExists{parskip.sty}{%
    \usepackage{parskip}
  }{% else
    \setlength{\parindent}{0pt}
    \setlength{\parskip}{6pt plus 2pt minus 1pt}}
}{% if KOMA class
  \KOMAoptions{parskip=half}}
\makeatother
\usepackage{xcolor}
\IfFileExists{xurl.sty}{\usepackage{xurl}}{} % add URL line breaks if available
\IfFileExists{bookmark.sty}{\usepackage{bookmark}}{\usepackage{hyperref}}
\hypersetup{
  pdftitle={2 데이터 구조},
  hidelinks,
  pdfcreator={LaTeX via pandoc}}
\urlstyle{same} % disable monospaced font for URLs
\usepackage[margin=1in]{geometry}
\usepackage{color}
\usepackage{fancyvrb}
\newcommand{\VerbBar}{|}
\newcommand{\VERB}{\Verb[commandchars=\\\{\}]}
\DefineVerbatimEnvironment{Highlighting}{Verbatim}{commandchars=\\\{\}}
% Add ',fontsize=\small' for more characters per line
\usepackage{framed}
\definecolor{shadecolor}{RGB}{248,248,248}
\newenvironment{Shaded}{\begin{snugshade}}{\end{snugshade}}
\newcommand{\AlertTok}[1]{\textcolor[rgb]{0.94,0.16,0.16}{#1}}
\newcommand{\AnnotationTok}[1]{\textcolor[rgb]{0.56,0.35,0.01}{\textbf{\textit{#1}}}}
\newcommand{\AttributeTok}[1]{\textcolor[rgb]{0.77,0.63,0.00}{#1}}
\newcommand{\BaseNTok}[1]{\textcolor[rgb]{0.00,0.00,0.81}{#1}}
\newcommand{\BuiltInTok}[1]{#1}
\newcommand{\CharTok}[1]{\textcolor[rgb]{0.31,0.60,0.02}{#1}}
\newcommand{\CommentTok}[1]{\textcolor[rgb]{0.56,0.35,0.01}{\textit{#1}}}
\newcommand{\CommentVarTok}[1]{\textcolor[rgb]{0.56,0.35,0.01}{\textbf{\textit{#1}}}}
\newcommand{\ConstantTok}[1]{\textcolor[rgb]{0.00,0.00,0.00}{#1}}
\newcommand{\ControlFlowTok}[1]{\textcolor[rgb]{0.13,0.29,0.53}{\textbf{#1}}}
\newcommand{\DataTypeTok}[1]{\textcolor[rgb]{0.13,0.29,0.53}{#1}}
\newcommand{\DecValTok}[1]{\textcolor[rgb]{0.00,0.00,0.81}{#1}}
\newcommand{\DocumentationTok}[1]{\textcolor[rgb]{0.56,0.35,0.01}{\textbf{\textit{#1}}}}
\newcommand{\ErrorTok}[1]{\textcolor[rgb]{0.64,0.00,0.00}{\textbf{#1}}}
\newcommand{\ExtensionTok}[1]{#1}
\newcommand{\FloatTok}[1]{\textcolor[rgb]{0.00,0.00,0.81}{#1}}
\newcommand{\FunctionTok}[1]{\textcolor[rgb]{0.00,0.00,0.00}{#1}}
\newcommand{\ImportTok}[1]{#1}
\newcommand{\InformationTok}[1]{\textcolor[rgb]{0.56,0.35,0.01}{\textbf{\textit{#1}}}}
\newcommand{\KeywordTok}[1]{\textcolor[rgb]{0.13,0.29,0.53}{\textbf{#1}}}
\newcommand{\NormalTok}[1]{#1}
\newcommand{\OperatorTok}[1]{\textcolor[rgb]{0.81,0.36,0.00}{\textbf{#1}}}
\newcommand{\OtherTok}[1]{\textcolor[rgb]{0.56,0.35,0.01}{#1}}
\newcommand{\PreprocessorTok}[1]{\textcolor[rgb]{0.56,0.35,0.01}{\textit{#1}}}
\newcommand{\RegionMarkerTok}[1]{#1}
\newcommand{\SpecialCharTok}[1]{\textcolor[rgb]{0.00,0.00,0.00}{#1}}
\newcommand{\SpecialStringTok}[1]{\textcolor[rgb]{0.31,0.60,0.02}{#1}}
\newcommand{\StringTok}[1]{\textcolor[rgb]{0.31,0.60,0.02}{#1}}
\newcommand{\VariableTok}[1]{\textcolor[rgb]{0.00,0.00,0.00}{#1}}
\newcommand{\VerbatimStringTok}[1]{\textcolor[rgb]{0.31,0.60,0.02}{#1}}
\newcommand{\WarningTok}[1]{\textcolor[rgb]{0.56,0.35,0.01}{\textbf{\textit{#1}}}}
\usepackage{graphicx}
\makeatletter
\def\maxwidth{\ifdim\Gin@nat@width>\linewidth\linewidth\else\Gin@nat@width\fi}
\def\maxheight{\ifdim\Gin@nat@height>\textheight\textheight\else\Gin@nat@height\fi}
\makeatother
% Scale images if necessary, so that they will not overflow the page
% margins by default, and it is still possible to overwrite the defaults
% using explicit options in \includegraphics[width, height, ...]{}
\setkeys{Gin}{width=\maxwidth,height=\maxheight,keepaspectratio}
% Set default figure placement to htbp
\makeatletter
\def\fps@figure{htbp}
\makeatother
\setlength{\emergencystretch}{3em} % prevent overfull lines
\providecommand{\tightlist}{%
  \setlength{\itemsep}{0pt}\setlength{\parskip}{0pt}}
\setcounter{secnumdepth}{-\maxdimen} % remove section numbering
\ifLuaTeX
  \usepackage{selnolig}  % disable illegal ligatures
\fi

\title{2 데이터 구조}
\author{}
\date{\vspace{-2.5em}}

\begin{document}
\maketitle

\hypertarget{uxbca1uxd130}{%
\subsection{2.1 벡터}\label{uxbca1uxd130}}

벡터(Vector)는 하나 이상의 데이터를 저장할 수 있는 1차원 저장 구조(1차원
배열)

\begin{Shaded}
\begin{Highlighting}[]
\NormalTok{students\_age }\OtherTok{\textless{}{-}} \FunctionTok{c}\NormalTok{(}\DecValTok{11}\NormalTok{, }\DecValTok{12}\NormalTok{, }\DecValTok{13}\NormalTok{, }\DecValTok{20}\NormalTok{, }\DecValTok{15}\NormalTok{, }\DecValTok{21}\NormalTok{)}
\NormalTok{students\_age}
\end{Highlighting}
\end{Shaded}

\begin{verbatim}
## [1] 11 12 13 20 15 21
\end{verbatim}

\begin{Shaded}
\begin{Highlighting}[]
\FunctionTok{class}\NormalTok{(students\_age)   }\CommentTok{\# 데이터 타입 확인}
\end{Highlighting}
\end{Shaded}

\begin{verbatim}
## [1] "numeric"
\end{verbatim}

\begin{Shaded}
\begin{Highlighting}[]
\FunctionTok{length}\NormalTok{(students\_age)  }\CommentTok{\# 벡터의 길이 확인}
\end{Highlighting}
\end{Shaded}

\begin{verbatim}
## [1] 6
\end{verbatim}

\begin{Shaded}
\begin{Highlighting}[]
\FunctionTok{str}\NormalTok{(students\_age)     }\CommentTok{\# 데이터 타입, 길이 등 전체 구조 확인}
\end{Highlighting}
\end{Shaded}

\begin{verbatim}
##  num [1:6] 11 12 13 20 15 21
\end{verbatim}

\hypertarget{uxc77cuxbd80-uxb370uxc774uxd130uxb9cc-uxc811uxadfc}{%
\subsubsection{2.1.1 일부 데이터만
접근}\label{uxc77cuxbd80-uxb370uxc774uxd130uxb9cc-uxc811uxadfc}}

인덱싱과 슬라이싱을 이용하여 일부 데이터만 접근할 수 있다.

\begin{Shaded}
\begin{Highlighting}[]
\CommentTok{\# 벡터 인덱싱(R의 인덱스는 1부터 시작)}
\NormalTok{students\_age[}\DecValTok{1}\NormalTok{]  }\CommentTok{\# 1번 인덱스의 데이터 추출하기}
\end{Highlighting}
\end{Shaded}

\begin{verbatim}
## [1] 11
\end{verbatim}

\begin{Shaded}
\begin{Highlighting}[]
\NormalTok{students\_age[}\DecValTok{3}\NormalTok{]  }\CommentTok{\# 3번 인덱스의 데이터 추출하기}
\end{Highlighting}
\end{Shaded}

\begin{verbatim}
## [1] 13
\end{verbatim}

\begin{Shaded}
\begin{Highlighting}[]
\NormalTok{students\_age[}\SpecialCharTok{{-}}\DecValTok{1}\NormalTok{]  }\CommentTok{\# 1번 인덱스의 데이터만 제외하고 추출하기기}
\end{Highlighting}
\end{Shaded}

\begin{verbatim}
## [1] 12 13 20 15 21
\end{verbatim}

\begin{Shaded}
\begin{Highlighting}[]
\CommentTok{\# 벡터 슬라이싱}
\NormalTok{students\_age[}\DecValTok{1}\SpecialCharTok{:}\DecValTok{3}\NormalTok{]  }\CommentTok{\# 1번부터 3번 인덱스까지의 데이터 추출하기}
\end{Highlighting}
\end{Shaded}

\begin{verbatim}
## [1] 11 12 13
\end{verbatim}

\begin{Shaded}
\begin{Highlighting}[]
\NormalTok{students\_age[}\DecValTok{4}\SpecialCharTok{:}\DecValTok{6}\NormalTok{]  }\CommentTok{\# 4번부터 6번 인덱스까지의 데이터 추출하기}
\end{Highlighting}
\end{Shaded}

\begin{verbatim}
## [1] 20 15 21
\end{verbatim}

\hypertarget{uxbca1uxd130uxc5d0-uxb370uxc774uxd130-uxcd94uxac00-uxac31uxc2e0}{%
\subsubsection{2.1.2 벡터에 데이터 추가,
갱신}\label{uxbca1uxd130uxc5d0-uxb370uxc774uxd130-uxcd94uxac00-uxac31uxc2e0}}

인덱스를 이용해서 벡터에 데이터를 갱신하거나 추가할 수 있다.

\begin{Shaded}
\begin{Highlighting}[]
\NormalTok{score }\OtherTok{\textless{}{-}} \FunctionTok{c}\NormalTok{(}\DecValTok{1}\NormalTok{,}\DecValTok{2}\NormalTok{,}\DecValTok{3}\NormalTok{)}
\NormalTok{score[}\DecValTok{1}\NormalTok{] }\OtherTok{\textless{}{-}} \DecValTok{10}
\NormalTok{score[}\DecValTok{4}\NormalTok{] }\OtherTok{\textless{}{-}} \DecValTok{4}
\NormalTok{score}
\end{Highlighting}
\end{Shaded}

\begin{verbatim}
## [1] 10  2  3  4
\end{verbatim}

\hypertarget{uxbca1uxd130uxc758-uxb370uxc774uxd130-uxd0c0uxc785}{%
\subsubsection{2.1.3 벡터의 데이터
타입}\label{uxbca1uxd130uxc758-uxb370uxc774uxd130-uxd0c0uxc785}}

벡터는 하나의 원시 데이터 타입으로 저장되므로, 다양한 데이터 타입을 섞어
저장하면 하나의 타입으로 자동 형변환된다.

\begin{Shaded}
\begin{Highlighting}[]
\NormalTok{code }\OtherTok{\textless{}{-}} \FunctionTok{c}\NormalTok{(}\DecValTok{1}\NormalTok{,}\DecValTok{12}\NormalTok{,}\StringTok{"30"}\NormalTok{)  }\CommentTok{\# 문자(character) 데이터 타입으로 모두 변환됨}
\FunctionTok{class}\NormalTok{(code) }
\end{Highlighting}
\end{Shaded}

\begin{verbatim}
## [1] "character"
\end{verbatim}

\begin{Shaded}
\begin{Highlighting}[]
\FunctionTok{str}\NormalTok{(code)}
\end{Highlighting}
\end{Shaded}

\begin{verbatim}
##  chr [1:3] "1" "12" "30"
\end{verbatim}

\begin{Shaded}
\begin{Highlighting}[]
\NormalTok{code }\OtherTok{\textless{}{-}} \FunctionTok{c}\NormalTok{(}\DecValTok{1}\NormalTok{,}\DecValTok{12}\NormalTok{, }\ConstantTok{TRUE}\NormalTok{, }\ConstantTok{FALSE}\NormalTok{)  }\CommentTok{\# 숫자(numeric) 데이터 타입으로 모두 변환됨}
\FunctionTok{class}\NormalTok{(code) }
\end{Highlighting}
\end{Shaded}

\begin{verbatim}
## [1] "numeric"
\end{verbatim}

\begin{Shaded}
\begin{Highlighting}[]
\FunctionTok{str}\NormalTok{(code)                     }\CommentTok{\# TRUE는 1, FALSE는 0의 숫자로 변환됨}
\end{Highlighting}
\end{Shaded}

\begin{verbatim}
##  num [1:4] 1 12 1 0
\end{verbatim}

\hypertarget{uxbca1uxd130-uxb370uxc774uxd130-uxc0dduxc131}{%
\subsubsection{2.1.3 벡터 데이터
생성}\label{uxbca1uxd130-uxb370uxc774uxd130-uxc0dduxc131}}

\begin{Shaded}
\begin{Highlighting}[]
\NormalTok{data }\OtherTok{\textless{}{-}} \FunctionTok{c}\NormalTok{(}\DecValTok{1}\SpecialCharTok{:}\DecValTok{10}\NormalTok{)  }\CommentTok{\# 1부터 10까지 1씩 증가시켜 생성하기}
\NormalTok{data}
\end{Highlighting}
\end{Shaded}

\begin{verbatim}
##  [1]  1  2  3  4  5  6  7  8  9 10
\end{verbatim}

\begin{Shaded}
\begin{Highlighting}[]
\NormalTok{data1 }\OtherTok{\textless{}{-}} \FunctionTok{seq}\NormalTok{(}\DecValTok{1}\NormalTok{,}\DecValTok{10}\NormalTok{) }\CommentTok{\# 1부터 10까지 1씩 증가시켜 생성하기}
\NormalTok{data1}
\end{Highlighting}
\end{Shaded}

\begin{verbatim}
##  [1]  1  2  3  4  5  6  7  8  9 10
\end{verbatim}

\begin{Shaded}
\begin{Highlighting}[]
\NormalTok{data2 }\OtherTok{\textless{}{-}} \FunctionTok{seq}\NormalTok{(}\DecValTok{1}\NormalTok{, }\DecValTok{10}\NormalTok{, }\AttributeTok{by=}\DecValTok{2}\NormalTok{) }\CommentTok{\# 1부터 10까지 2씩 증가시켜 생성하기}
\NormalTok{data2}
\end{Highlighting}
\end{Shaded}

\begin{verbatim}
## [1] 1 3 5 7 9
\end{verbatim}

\begin{Shaded}
\begin{Highlighting}[]
\NormalTok{data3 }\OtherTok{\textless{}{-}} \FunctionTok{rep}\NormalTok{(}\DecValTok{1}\NormalTok{, }\AttributeTok{times=}\DecValTok{5}\NormalTok{)  }\CommentTok{\#1을 다섯 번 반복 생성하기}
\NormalTok{data3}
\end{Highlighting}
\end{Shaded}

\begin{verbatim}
## [1] 1 1 1 1 1
\end{verbatim}

\begin{Shaded}
\begin{Highlighting}[]
\NormalTok{data4 }\OtherTok{\textless{}{-}} \FunctionTok{rep}\NormalTok{(}\DecValTok{1}\SpecialCharTok{:}\DecValTok{3}\NormalTok{, }\AttributeTok{each=}\DecValTok{3}\NormalTok{)  }\CommentTok{\#1부터 3을 각각 세 번씩 반복 생성하기}
\NormalTok{data4}
\end{Highlighting}
\end{Shaded}

\begin{verbatim}
## [1] 1 1 1 2 2 2 3 3 3
\end{verbatim}

\begin{Shaded}
\begin{Highlighting}[]
\NormalTok{data5 }\OtherTok{\textless{}{-}} \FunctionTok{rep}\NormalTok{(}\DecValTok{1}\SpecialCharTok{:}\DecValTok{3}\NormalTok{, }\AttributeTok{times=}\DecValTok{3}\NormalTok{)}
\NormalTok{data5}
\end{Highlighting}
\end{Shaded}

\begin{verbatim}
## [1] 1 2 3 1 2 3 1 2 3
\end{verbatim}

\begin{center}\rule{0.5\linewidth}{0.5pt}\end{center}

\hypertarget{uxd589uxb82c}{%
\subsection{2.2 행렬}\label{uxd589uxb82c}}

행렬(Matrix)는 표 형태와 같은 2차원 데이터 저장 구조를 가진다. 행렬은
벡터와 마찬가지로 모두 같은 데이터 타입이어야 한다.

\begin{Shaded}
\begin{Highlighting}[]
\NormalTok{var1 }\OtherTok{\textless{}{-}} \FunctionTok{c}\NormalTok{(}\DecValTok{1}\NormalTok{, }\DecValTok{2}\NormalTok{, }\DecValTok{3}\NormalTok{, }\DecValTok{4}\NormalTok{, }\DecValTok{5}\NormalTok{, }\DecValTok{6}\NormalTok{)}

\CommentTok{\# var1을 이용해서 2행 3열 행렬을 생성. 기본적으로 열 우선으로 값이 채워짐}
\NormalTok{x1 }\OtherTok{\textless{}{-}} \FunctionTok{matrix}\NormalTok{(var1, }\AttributeTok{nrow=}\DecValTok{2}\NormalTok{, }\AttributeTok{ncol=}\DecValTok{3}\NormalTok{)}
\NormalTok{x1}
\end{Highlighting}
\end{Shaded}

\begin{verbatim}
##      [,1] [,2] [,3]
## [1,]    1    3    5
## [2,]    2    4    6
\end{verbatim}

\begin{Shaded}
\begin{Highlighting}[]
\CommentTok{\# var1을 이용해서 2열 행렬을 생성. 행의 개수는 자동 계산됨}
\NormalTok{x2 }\OtherTok{\textless{}{-}} \FunctionTok{matrix}\NormalTok{(var1, }\AttributeTok{ncol=}\DecValTok{2}\NormalTok{)}
\NormalTok{x2}
\end{Highlighting}
\end{Shaded}

\begin{verbatim}
##      [,1] [,2]
## [1,]    1    4
## [2,]    2    5
## [3,]    3    6
\end{verbatim}

\hypertarget{uxc77cuxbd80-uxb370uxc774uxd130uxb9cc-uxc811uxadfc-1}{%
\subsubsection{2.2.1 일부 데이터만
접근}\label{uxc77cuxbd80-uxb370uxc774uxd130uxb9cc-uxc811uxadfc-1}}

\begin{Shaded}
\begin{Highlighting}[]
\NormalTok{x1[}\DecValTok{1}\NormalTok{,]  }\CommentTok{\# x1의 1행, 모든 열}
\end{Highlighting}
\end{Shaded}

\begin{verbatim}
## [1] 1 3 5
\end{verbatim}

\begin{Shaded}
\begin{Highlighting}[]
\NormalTok{x1[,}\DecValTok{1}\NormalTok{]  }\CommentTok{\# x1의 모든 행, 1열}
\end{Highlighting}
\end{Shaded}

\begin{verbatim}
## [1] 1 2
\end{verbatim}

\begin{Shaded}
\begin{Highlighting}[]
\NormalTok{x1[}\DecValTok{2}\NormalTok{,}\DecValTok{2}\NormalTok{]  }\CommentTok{\#x1의 2행, 2열}
\end{Highlighting}
\end{Shaded}

\begin{verbatim}
## [1] 4
\end{verbatim}

dimnames()로 행렬의 행 이름, 열 이름을 부여할 수 있으며, 행 이름과 열
이름으로도 데이터 접근이 가능하다.

\begin{Shaded}
\begin{Highlighting}[]
\NormalTok{var1 }\OtherTok{\textless{}{-}} \FunctionTok{c}\NormalTok{(}\DecValTok{1}\NormalTok{, }\DecValTok{2}\NormalTok{, }\DecValTok{3}\NormalTok{, }\DecValTok{4}\NormalTok{, }\DecValTok{5}\NormalTok{, }\DecValTok{6}\NormalTok{)}
\NormalTok{x2 }\OtherTok{\textless{}{-}} \FunctionTok{matrix}\NormalTok{(var1, }\AttributeTok{ncol=}\DecValTok{2}\NormalTok{)}
\FunctionTok{dimnames}\NormalTok{(x2) }\OtherTok{\textless{}{-}} \FunctionTok{list}\NormalTok{(}\FunctionTok{c}\NormalTok{(}\StringTok{"r1"}\NormalTok{, }\StringTok{"r2"}\NormalTok{, }\StringTok{"r3"}\NormalTok{), }\FunctionTok{c}\NormalTok{(}\StringTok{"c1"}\NormalTok{, }\StringTok{"c2"}\NormalTok{))   }\CommentTok{\# x2행렬에 행 이름과 열 이름 부여}
\NormalTok{x2}
\end{Highlighting}
\end{Shaded}

\begin{verbatim}
##    c1 c2
## r1  1  4
## r2  2  5
## r3  3  6
\end{verbatim}

\begin{Shaded}
\begin{Highlighting}[]
\NormalTok{x2[, }\StringTok{"c1"}\NormalTok{]      }\CommentTok{\# x2의 모든 행, c1열}
\end{Highlighting}
\end{Shaded}

\begin{verbatim}
## r1 r2 r3 
##  1  2  3
\end{verbatim}

\begin{Shaded}
\begin{Highlighting}[]
\NormalTok{x2[}\StringTok{"r1"}\NormalTok{, ]      }\CommentTok{\# x2의 r1행, 모든 열}
\end{Highlighting}
\end{Shaded}

\begin{verbatim}
## c1 c2 
##  1  4
\end{verbatim}

\begin{Shaded}
\begin{Highlighting}[]
\NormalTok{x2[}\StringTok{"r1"}\NormalTok{, }\StringTok{"c1"}\NormalTok{]  }\CommentTok{\# x2의 r1행, c1열}
\end{Highlighting}
\end{Shaded}

\begin{verbatim}
## [1] 1
\end{verbatim}

\hypertarget{uxd589uxb82cuxc5d0-uxb370uxc774uxd130-uxcd94uxac00}{%
\subsubsection{2.2.2 행렬에 데이터
추가}\label{uxd589uxb82cuxc5d0-uxb370uxc774uxd130-uxcd94uxac00}}

\begin{Shaded}
\begin{Highlighting}[]
\NormalTok{x1 }\OtherTok{\textless{}{-}} \FunctionTok{matrix}\NormalTok{(}\FunctionTok{c}\NormalTok{(}\DecValTok{1}\NormalTok{, }\DecValTok{2}\NormalTok{, }\DecValTok{3}\NormalTok{, }\DecValTok{4}\NormalTok{, }\DecValTok{5}\NormalTok{, }\DecValTok{6}\NormalTok{), }\AttributeTok{nrow=}\DecValTok{2}\NormalTok{, }\AttributeTok{ncol=}\DecValTok{3}\NormalTok{)}
\NormalTok{x1}
\end{Highlighting}
\end{Shaded}

\begin{verbatim}
##      [,1] [,2] [,3]
## [1,]    1    3    5
## [2,]    2    4    6
\end{verbatim}

\begin{Shaded}
\begin{Highlighting}[]
\NormalTok{x1 }\OtherTok{\textless{}{-}} \FunctionTok{rbind}\NormalTok{(x1, }\FunctionTok{c}\NormalTok{(}\DecValTok{10}\NormalTok{, }\DecValTok{10}\NormalTok{, }\DecValTok{10}\NormalTok{))  }\CommentTok{\# 행 추가}
\NormalTok{x1 }\OtherTok{\textless{}{-}} \FunctionTok{cbind}\NormalTok{(x1, }\FunctionTok{c}\NormalTok{(}\DecValTok{20}\NormalTok{, }\DecValTok{20}\NormalTok{, }\DecValTok{20}\NormalTok{))  }\CommentTok{\# 열 추가}
\NormalTok{x1}
\end{Highlighting}
\end{Shaded}

\begin{verbatim}
##      [,1] [,2] [,3] [,4]
## [1,]    1    3    5   20
## [2,]    2    4    6   20
## [3,]   10   10   10   20
\end{verbatim}

\begin{center}\rule{0.5\linewidth}{0.5pt}\end{center}

\hypertarget{uxb370uxc774uxd130uxd504uxb808uxc784}{%
\subsection{2.3
데이터프레임}\label{uxb370uxc774uxd130uxd504uxb808uxc784}}

데이터프레임(Dataframe)은 행렬처럼 행과 열을 가진 2차원 구조다.
\textbf{벡터, 행렬과 다른 점은 각 열이 서로 다른 데이터 형식을 가질 수
있다}는 것이다(단, 각 벡터의 길이가 동일해야 한다).

\begin{Shaded}
\begin{Highlighting}[]
\NormalTok{no }\OtherTok{\textless{}{-}} \FunctionTok{c}\NormalTok{(}\DecValTok{10}\NormalTok{,}\DecValTok{20}\NormalTok{,}\DecValTok{30}\NormalTok{,}\DecValTok{40}\NormalTok{,}\DecValTok{50}\NormalTok{,}\DecValTok{60}\NormalTok{,}\DecValTok{70}\NormalTok{)}
\NormalTok{age }\OtherTok{\textless{}{-}} \FunctionTok{c}\NormalTok{(}\DecValTok{18}\NormalTok{,}\DecValTok{15}\NormalTok{,}\DecValTok{13}\NormalTok{,}\DecValTok{12}\NormalTok{,}\DecValTok{10}\NormalTok{,}\DecValTok{9}\NormalTok{,}\DecValTok{7}\NormalTok{)}
\NormalTok{gender }\OtherTok{\textless{}{-}} \FunctionTok{c}\NormalTok{(}\StringTok{"M"}\NormalTok{, }\StringTok{"M"}\NormalTok{, }\StringTok{"M"}\NormalTok{, }\StringTok{"M"}\NormalTok{, }\StringTok{"M"}\NormalTok{, }\StringTok{"F"}\NormalTok{, }\StringTok{"M"}\NormalTok{)}

\CommentTok{\# 데이터프레임 생성}
\NormalTok{students }\OtherTok{\textless{}{-}} \FunctionTok{data.frame}\NormalTok{(no, age, gender)}
\NormalTok{students}
\end{Highlighting}
\end{Shaded}

\begin{verbatim}
##   no age gender
## 1 10  18      M
## 2 20  15      M
## 3 30  13      M
## 4 40  12      M
## 5 50  10      M
## 6 60   9      F
## 7 70   7      M
\end{verbatim}

\begin{Shaded}
\begin{Highlighting}[]
\CommentTok{\# 열의 이름과 행의 이름 확인}
\FunctionTok{colnames}\NormalTok{(students)  }\CommentTok{\# 열 이름 확인}
\end{Highlighting}
\end{Shaded}

\begin{verbatim}
## [1] "no"     "age"    "gender"
\end{verbatim}

\begin{Shaded}
\begin{Highlighting}[]
\FunctionTok{rownames}\NormalTok{(students)  }\CommentTok{\# 행 이름 확인}
\end{Highlighting}
\end{Shaded}

\begin{verbatim}
## [1] "1" "2" "3" "4" "5" "6" "7"
\end{verbatim}

\begin{Shaded}
\begin{Highlighting}[]
\FunctionTok{colnames}\NormalTok{(students)  }\OtherTok{\textless{}{-}} \FunctionTok{c}\NormalTok{(}\StringTok{"no"}\NormalTok{, }\StringTok{"나이"}\NormalTok{, }\StringTok{"성별"}\NormalTok{)               }\CommentTok{\# 열 이름 수정}
\FunctionTok{rownames}\NormalTok{(students)  }\OtherTok{\textless{}{-}} \FunctionTok{c}\NormalTok{(}\StringTok{\textquotesingle{}A\textquotesingle{}}\NormalTok{, }\StringTok{\textquotesingle{}B\textquotesingle{}}\NormalTok{, }\StringTok{\textquotesingle{}C\textquotesingle{}}\NormalTok{, }\StringTok{\textquotesingle{}D\textquotesingle{}}\NormalTok{, }\StringTok{\textquotesingle{}E\textquotesingle{}}\NormalTok{, }\StringTok{\textquotesingle{}F\textquotesingle{}}\NormalTok{, }\StringTok{\textquotesingle{}G\textquotesingle{}}\NormalTok{)  }\CommentTok{\# 행 이름 수정}

\NormalTok{students}
\end{Highlighting}
\end{Shaded}

\begin{verbatim}
##   no 나이 성별
## A 10   18    M
## B 20   15    M
## C 30   13    M
## D 40   12    M
## E 50   10    M
## F 60    9    F
## G 70    7    M
\end{verbatim}

\hypertarget{uxc77cuxbd80-uxb370uxc774uxd130uxb9cc-uxc811uxadfc-2}{%
\subsubsection{2.3.1 일부 데이터만
접근}\label{uxc77cuxbd80-uxb370uxc774uxd130uxb9cc-uxc811uxadfc-2}}

\begin{Shaded}
\begin{Highlighting}[]
\NormalTok{no }\OtherTok{\textless{}{-}} \FunctionTok{c}\NormalTok{(}\DecValTok{10}\NormalTok{,}\DecValTok{20}\NormalTok{,}\DecValTok{30}\NormalTok{,}\DecValTok{40}\NormalTok{,}\DecValTok{50}\NormalTok{,}\DecValTok{60}\NormalTok{,}\DecValTok{70}\NormalTok{)}
\NormalTok{age }\OtherTok{\textless{}{-}} \FunctionTok{c}\NormalTok{(}\DecValTok{18}\NormalTok{,}\DecValTok{15}\NormalTok{,}\DecValTok{13}\NormalTok{,}\DecValTok{12}\NormalTok{,}\DecValTok{10}\NormalTok{,}\DecValTok{9}\NormalTok{,}\DecValTok{7}\NormalTok{)}
\NormalTok{gender }\OtherTok{\textless{}{-}} \FunctionTok{c}\NormalTok{(}\StringTok{"M"}\NormalTok{, }\StringTok{"M"}\NormalTok{, }\StringTok{"M"}\NormalTok{, }\StringTok{"M"}\NormalTok{, }\StringTok{"M"}\NormalTok{, }\StringTok{"F"}\NormalTok{, }\StringTok{"M"}\NormalTok{)}
\NormalTok{students }\OtherTok{\textless{}{-}} \FunctionTok{data.frame}\NormalTok{(no, age, gender)}
\FunctionTok{rownames}\NormalTok{(students)  }\OtherTok{\textless{}{-}} \FunctionTok{c}\NormalTok{(}\StringTok{\textquotesingle{}A\textquotesingle{}}\NormalTok{, }\StringTok{\textquotesingle{}B\textquotesingle{}}\NormalTok{, }\StringTok{\textquotesingle{}C\textquotesingle{}}\NormalTok{, }\StringTok{\textquotesingle{}D\textquotesingle{}}\NormalTok{, }\StringTok{\textquotesingle{}E\textquotesingle{}}\NormalTok{, }\StringTok{\textquotesingle{}F\textquotesingle{}}\NormalTok{, }\StringTok{\textquotesingle{}G\textquotesingle{}}\NormalTok{)}

\CommentTok{\# 열 이름으로 특정 열에 접근하기}
\CommentTok{\# 데이터프레임의 변수이름$열이름으로 특정 열에 접근하기}
\NormalTok{students}\SpecialCharTok{$}\NormalTok{no}
\end{Highlighting}
\end{Shaded}

\begin{verbatim}
## [1] 10 20 30 40 50 60 70
\end{verbatim}

\begin{Shaded}
\begin{Highlighting}[]
\NormalTok{students}\SpecialCharTok{$}\NormalTok{age}
\end{Highlighting}
\end{Shaded}

\begin{verbatim}
## [1] 18 15 13 12 10  9  7
\end{verbatim}

\begin{Shaded}
\begin{Highlighting}[]
\CommentTok{\# 대괄호 안에 열이름으로 특정 열에 접근하기}
\CommentTok{\# 대괄호 안에 콤마(,)를 쓴 후 열이름을 쓴다. 열이름은 " " 또는 \textquotesingle{} \textquotesingle{}로 감쌈}
\NormalTok{students[,}\StringTok{"no"}\NormalTok{]}
\end{Highlighting}
\end{Shaded}

\begin{verbatim}
## [1] 10 20 30 40 50 60 70
\end{verbatim}

\begin{Shaded}
\begin{Highlighting}[]
\NormalTok{students[,}\StringTok{"age"}\NormalTok{]}
\end{Highlighting}
\end{Shaded}

\begin{verbatim}
## [1] 18 15 13 12 10  9  7
\end{verbatim}

\begin{Shaded}
\begin{Highlighting}[]
\CommentTok{\# 열 인덱스로 특정 열에 접근하기}
\NormalTok{students[,}\DecValTok{1}\NormalTok{]  }\CommentTok{\# 첫번째 열 데이터가 모두 출력됨}
\end{Highlighting}
\end{Shaded}

\begin{verbatim}
## [1] 10 20 30 40 50 60 70
\end{verbatim}

\begin{Shaded}
\begin{Highlighting}[]
\NormalTok{students[,}\DecValTok{2}\NormalTok{]  }\CommentTok{\# 두번째 열 데이터가 모두 출력됨}
\end{Highlighting}
\end{Shaded}

\begin{verbatim}
## [1] 18 15 13 12 10  9  7
\end{verbatim}

\begin{Shaded}
\begin{Highlighting}[]
\CommentTok{\# 행 이름으로 특정 행만 접근하기}
\NormalTok{students[}\StringTok{"A"}\NormalTok{, ]  }\CommentTok{\# A행 데이터가 출력된다. 행이름은 " " 또는 \textquotesingle{} \textquotesingle{}로 감쌈}
\end{Highlighting}
\end{Shaded}

\begin{verbatim}
##   no age gender
## A 10  18      M
\end{verbatim}

\begin{Shaded}
\begin{Highlighting}[]
                 \CommentTok{\# 행 이름 뒤에 콤마(,)를 반드시 써야함}

\CommentTok{\# 행 인덱스로 특정 행만 접근하기}
\NormalTok{students[}\DecValTok{2}\NormalTok{,]  }\CommentTok{\# 두번째 행 데이터가 출력}
\end{Highlighting}
\end{Shaded}

\begin{verbatim}
##   no age gender
## B 20  15      M
\end{verbatim}

\begin{Shaded}
\begin{Highlighting}[]
              \CommentTok{\# 행 인덱스 뒤에 콤마(,)를 반드시 써야함}

\CommentTok{\# 행 인덱스, 열 인덱스 또는 행 이름, 열 이름으로 데이터에 접근하기}
\NormalTok{students[}\DecValTok{3}\NormalTok{,}\DecValTok{1}\NormalTok{]        }\CommentTok{\# 변수이름[행인덱스, 열인덱스]로 작성}
\end{Highlighting}
\end{Shaded}

\begin{verbatim}
## [1] 30
\end{verbatim}

\begin{Shaded}
\begin{Highlighting}[]
\NormalTok{students[}\StringTok{"A"}\NormalTok{, }\StringTok{"no"}\NormalTok{]  }\CommentTok{\# 변수이름["행이름", "열이름"]으로 작성}
\end{Highlighting}
\end{Shaded}

\begin{verbatim}
## [1] 10
\end{verbatim}

\hypertarget{uxb370uxc774uxd130uxd504uxb808uxc784uxc758-uxb370uxc774uxd130-uxd0c0uxc785}{%
\subsubsection{2.3.2 데이터프레임의 데이터
타입}\label{uxb370uxc774uxd130uxd504uxb808uxc784uxc758-uxb370uxc774uxd130-uxd0c0uxc785}}

벡터의 경우 class()로 데이터 타입을 확인하면 벡터 내에 저장된 데이터
타입이 출력되지만 그 외의 데이터 구조는 데이터 구조 자체의 타입이
출력된다. 데이터 구조도 하나의 데이터 타입이다.

\begin{Shaded}
\begin{Highlighting}[]
\NormalTok{no }\OtherTok{\textless{}{-}} \FunctionTok{c}\NormalTok{(}\DecValTok{10}\NormalTok{,}\DecValTok{20}\NormalTok{,}\DecValTok{30}\NormalTok{,}\DecValTok{40}\NormalTok{,}\DecValTok{50}\NormalTok{,}\DecValTok{60}\NormalTok{,}\DecValTok{70}\NormalTok{)}
\NormalTok{age }\OtherTok{\textless{}{-}} \FunctionTok{c}\NormalTok{(}\DecValTok{18}\NormalTok{,}\DecValTok{15}\NormalTok{,}\DecValTok{13}\NormalTok{,}\DecValTok{12}\NormalTok{,}\DecValTok{10}\NormalTok{,}\DecValTok{9}\NormalTok{,}\DecValTok{7}\NormalTok{)}
\NormalTok{gender }\OtherTok{\textless{}{-}} \FunctionTok{c}\NormalTok{(}\StringTok{"M"}\NormalTok{, }\StringTok{"M"}\NormalTok{, }\StringTok{"M"}\NormalTok{, }\StringTok{"M"}\NormalTok{, }\StringTok{"M"}\NormalTok{, }\StringTok{"F"}\NormalTok{, }\StringTok{"M"}\NormalTok{)}
\NormalTok{students }\OtherTok{\textless{}{-}} \FunctionTok{data.frame}\NormalTok{(no, age, gender)}

\FunctionTok{class}\NormalTok{(students)}
\end{Highlighting}
\end{Shaded}

\begin{verbatim}
## [1] "data.frame"
\end{verbatim}

\begin{Shaded}
\begin{Highlighting}[]
\CommentTok{\# 특정 열의 데이터 타입 확인}
\FunctionTok{class}\NormalTok{(students}\SpecialCharTok{$}\NormalTok{no)}
\end{Highlighting}
\end{Shaded}

\begin{verbatim}
## [1] "numeric"
\end{verbatim}

\begin{Shaded}
\begin{Highlighting}[]
\FunctionTok{class}\NormalTok{(students}\SpecialCharTok{$}\NormalTok{gender)}
\end{Highlighting}
\end{Shaded}

\begin{verbatim}
## [1] "character"
\end{verbatim}

데이터프레임 생성 시 문자타입을 팩터타입으로 생성하고 싶으면
stringsAsFactors = TRUE 옵션을 사용하면 된다.

\begin{Shaded}
\begin{Highlighting}[]
\NormalTok{no }\OtherTok{\textless{}{-}} \FunctionTok{c}\NormalTok{(}\DecValTok{10}\NormalTok{,}\DecValTok{20}\NormalTok{,}\DecValTok{30}\NormalTok{,}\DecValTok{40}\NormalTok{,}\DecValTok{50}\NormalTok{,}\DecValTok{60}\NormalTok{,}\DecValTok{70}\NormalTok{)}
\NormalTok{age }\OtherTok{\textless{}{-}} \FunctionTok{c}\NormalTok{(}\DecValTok{18}\NormalTok{,}\DecValTok{15}\NormalTok{,}\DecValTok{13}\NormalTok{,}\DecValTok{12}\NormalTok{,}\DecValTok{10}\NormalTok{,}\DecValTok{9}\NormalTok{,}\DecValTok{7}\NormalTok{)}
\NormalTok{gender }\OtherTok{\textless{}{-}} \FunctionTok{c}\NormalTok{(}\StringTok{"M"}\NormalTok{, }\StringTok{"M"}\NormalTok{, }\StringTok{"M"}\NormalTok{, }\StringTok{"M"}\NormalTok{, }\StringTok{"M"}\NormalTok{, }\StringTok{"F"}\NormalTok{, }\StringTok{"M"}\NormalTok{)}

\NormalTok{students }\OtherTok{\textless{}{-}} \FunctionTok{data.frame}\NormalTok{(no, age, gender, }\AttributeTok{stringsAsFactors =} \ConstantTok{TRUE}\NormalTok{)}

\FunctionTok{class}\NormalTok{(students}\SpecialCharTok{$}\NormalTok{gender)}
\end{Highlighting}
\end{Shaded}

\begin{verbatim}
## [1] "factor"
\end{verbatim}

\begin{quote}
데이터 타입을 확인하기 위해 class(), typeof(), mode() /등의 함수를
사용할 수 있다. 세 함수는 각각 다른 값을 가질 수도 아닐 수도 있다.
class()는 객체지향 관덤에서 상속받은 클래스 이름을 반환하고, typeof()는
내부에 저장되는 형식인 원시 데이터 타입을 반환한다. mode()는
typeof()ㅣ보다 더 넓은 의미의 데이터 타입이다. 예를 들어 typeof()가
double이나 interger로 반환하는 타입을 mode()는 모두 numeric으로
반환한다.
\end{quote}

\begin{Shaded}
\begin{Highlighting}[]
\NormalTok{a }\OtherTok{\textless{}{-}} \DecValTok{10}
\FunctionTok{class}\NormalTok{(a)}
\end{Highlighting}
\end{Shaded}

\begin{verbatim}
## [1] "numeric"
\end{verbatim}

\begin{Shaded}
\begin{Highlighting}[]
\FunctionTok{mode}\NormalTok{(a)}
\end{Highlighting}
\end{Shaded}

\begin{verbatim}
## [1] "numeric"
\end{verbatim}

\begin{Shaded}
\begin{Highlighting}[]
\FunctionTok{typeof}\NormalTok{(a)}
\end{Highlighting}
\end{Shaded}

\begin{verbatim}
## [1] "double"
\end{verbatim}

\hypertarget{uxb370uxc774uxd130uxd504uxb808uxc784uxc758-uxad6cuxc870}{%
\subsubsection{2.3.3 데이터프레임의
구조}\label{uxb370uxc774uxd130uxd504uxb808uxc784uxc758-uxad6cuxc870}}

\begin{Shaded}
\begin{Highlighting}[]
\NormalTok{no }\OtherTok{\textless{}{-}} \FunctionTok{c}\NormalTok{(}\DecValTok{10}\NormalTok{,}\DecValTok{20}\NormalTok{,}\DecValTok{30}\NormalTok{,}\DecValTok{40}\NormalTok{,}\DecValTok{50}\NormalTok{,}\DecValTok{60}\NormalTok{,}\DecValTok{70}\NormalTok{)}
\NormalTok{age }\OtherTok{\textless{}{-}} \FunctionTok{c}\NormalTok{(}\DecValTok{18}\NormalTok{,}\DecValTok{15}\NormalTok{,}\DecValTok{13}\NormalTok{,}\DecValTok{12}\NormalTok{,}\DecValTok{10}\NormalTok{,}\DecValTok{9}\NormalTok{,}\DecValTok{7}\NormalTok{)}
\NormalTok{gender }\OtherTok{\textless{}{-}} \FunctionTok{c}\NormalTok{(}\StringTok{"M"}\NormalTok{, }\StringTok{"M"}\NormalTok{, }\StringTok{"M"}\NormalTok{, }\StringTok{"M"}\NormalTok{, }\StringTok{"M"}\NormalTok{, }\StringTok{"F"}\NormalTok{, }\StringTok{"M"}\NormalTok{)}
\NormalTok{students }\OtherTok{\textless{}{-}} \FunctionTok{data.frame}\NormalTok{(no, age, gender)}

\CommentTok{\# str()로 대략적 구조 확인}
\FunctionTok{str}\NormalTok{(students)}
\end{Highlighting}
\end{Shaded}

\begin{verbatim}
## 'data.frame':    7 obs. of  3 variables:
##  $ no    : num  10 20 30 40 50 60 70
##  $ age   : num  18 15 13 12 10 9 7
##  $ gender: chr  "M" "M" "M" "M" ...
\end{verbatim}

\begin{Shaded}
\begin{Highlighting}[]
\CommentTok{\# dim()으로 차원 정보 확인}
\FunctionTok{dim}\NormalTok{(students)}
\end{Highlighting}
\end{Shaded}

\begin{verbatim}
## [1] 7 3
\end{verbatim}

\begin{Shaded}
\begin{Highlighting}[]
\CommentTok{\# head()와 tail()로 일부 데이터만 추출 (빅데이터의 경우 유용)}
\FunctionTok{head}\NormalTok{(students)   }\CommentTok{\# 앞의 6행만 추출}
\end{Highlighting}
\end{Shaded}

\begin{verbatim}
##   no age gender
## 1 10  18      M
## 2 20  15      M
## 3 30  13      M
## 4 40  12      M
## 5 50  10      M
## 6 60   9      F
\end{verbatim}

\begin{Shaded}
\begin{Highlighting}[]
\FunctionTok{tail}\NormalTok{(students)   }\CommentTok{\# 뒤의 6행만 추출}
\end{Highlighting}
\end{Shaded}

\begin{verbatim}
##   no age gender
## 2 20  15      M
## 3 30  13      M
## 4 40  12      M
## 5 50  10      M
## 6 60   9      F
## 7 70   7      M
\end{verbatim}

\hypertarget{uxb370uxc774uxd130uxd504uxb808uxc784-uxb370uxc774uxd130-uxcd94uxac00}{%
\subsubsection{2.3.4 데이터프레임 데이터
추가}\label{uxb370uxc774uxd130uxd504uxb808uxc784-uxb370uxc774uxd130-uxcd94uxac00}}

\begin{Shaded}
\begin{Highlighting}[]
\NormalTok{no }\OtherTok{\textless{}{-}} \FunctionTok{c}\NormalTok{(}\DecValTok{10}\NormalTok{,}\DecValTok{20}\NormalTok{,}\DecValTok{30}\NormalTok{,}\DecValTok{40}\NormalTok{,}\DecValTok{50}\NormalTok{,}\DecValTok{60}\NormalTok{,}\DecValTok{70}\NormalTok{)}
\NormalTok{age }\OtherTok{\textless{}{-}} \FunctionTok{c}\NormalTok{(}\DecValTok{18}\NormalTok{,}\DecValTok{15}\NormalTok{,}\DecValTok{13}\NormalTok{,}\DecValTok{12}\NormalTok{,}\DecValTok{10}\NormalTok{,}\DecValTok{9}\NormalTok{,}\DecValTok{7}\NormalTok{)}
\NormalTok{gender }\OtherTok{\textless{}{-}} \FunctionTok{c}\NormalTok{(}\StringTok{"M"}\NormalTok{, }\StringTok{"M"}\NormalTok{, }\StringTok{"M"}\NormalTok{, }\StringTok{"M"}\NormalTok{, }\StringTok{"M"}\NormalTok{, }\StringTok{"F"}\NormalTok{, }\StringTok{"M"}\NormalTok{)}
\NormalTok{students }\OtherTok{\textless{}{-}} \FunctionTok{data.frame}\NormalTok{(no, age, gender)}
\FunctionTok{rownames}\NormalTok{(students)  }\OtherTok{\textless{}{-}} \FunctionTok{c}\NormalTok{(}\StringTok{\textquotesingle{}A\textquotesingle{}}\NormalTok{, }\StringTok{\textquotesingle{}B\textquotesingle{}}\NormalTok{, }\StringTok{\textquotesingle{}C\textquotesingle{}}\NormalTok{, }\StringTok{\textquotesingle{}D\textquotesingle{}}\NormalTok{, }\StringTok{\textquotesingle{}E\textquotesingle{}}\NormalTok{, }\StringTok{\textquotesingle{}F\textquotesingle{}}\NormalTok{, }\StringTok{\textquotesingle{}G\textquotesingle{}}\NormalTok{)}

\CommentTok{\# 열 데이터 추가}
\NormalTok{students}\SpecialCharTok{$}\NormalTok{name }\OtherTok{\textless{}{-}} \FunctionTok{c}\NormalTok{(}\StringTok{"이용"}\NormalTok{, }\StringTok{"준희"}\NormalTok{, }\StringTok{"이훈"}\NormalTok{, }\StringTok{"서희"}\NormalTok{, }\StringTok{"승희"}\NormalTok{, }\StringTok{"하정"}\NormalTok{, }\StringTok{"하준"}\NormalTok{)  }\CommentTok{\# 열 추가}
\NormalTok{students}
\end{Highlighting}
\end{Shaded}

\begin{verbatim}
##   no age gender name
## A 10  18      M 이용
## B 20  15      M 준희
## C 30  13      M 이훈
## D 40  12      M 서희
## E 50  10      M 승희
## F 60   9      F 하정
## G 70   7      M 하준
\end{verbatim}

\begin{Shaded}
\begin{Highlighting}[]
\CommentTok{\# 행 데이터 추가}
\NormalTok{students[}\StringTok{"H"}\NormalTok{,] }\OtherTok{\textless{}{-}} \FunctionTok{c}\NormalTok{(}\DecValTok{80}\NormalTok{,}\DecValTok{10}\NormalTok{,}\StringTok{\textquotesingle{}M\textquotesingle{}}\NormalTok{,}\StringTok{\textquotesingle{}홍길동\textquotesingle{}}\NormalTok{)  }\CommentTok{\# 헹 추가}
\FunctionTok{tail}\NormalTok{(students)}
\end{Highlighting}
\end{Shaded}

\begin{verbatim}
##   no age gender   name
## C 30  13      M   이훈
## D 40  12      M   서희
## E 50  10      M   승희
## F 60   9      F   하정
## G 70   7      M   하준
## H 80  10      M 홍길동
\end{verbatim}

\begin{center}\rule{0.5\linewidth}{0.5pt}\end{center}

\hypertarget{uxbc30uxc5f4}{%
\subsection{2.4 배열}\label{uxbc30uxc5f4}}

배열(Array)는 다차원 데이터 저장 구조다. 벡터나 행렬처럼 하나의 원시
데이터 타입으로 저장된다.

\begin{Shaded}
\begin{Highlighting}[]
\NormalTok{var1 }\OtherTok{\textless{}{-}} \FunctionTok{c}\NormalTok{(}\DecValTok{1}\SpecialCharTok{:}\DecValTok{12}\NormalTok{)  }\CommentTok{\# 벡터 생성하기}

\NormalTok{arr1 }\OtherTok{\textless{}{-}} \FunctionTok{array}\NormalTok{(var1, }\AttributeTok{dim=}\FunctionTok{c}\NormalTok{(}\DecValTok{2}\NormalTok{,}\DecValTok{2}\NormalTok{,}\DecValTok{3}\NormalTok{))    }\CommentTok{\# 3차원 배열 생성}
\NormalTok{arr1}
\end{Highlighting}
\end{Shaded}

\begin{verbatim}
## , , 1
## 
##      [,1] [,2]
## [1,]    1    3
## [2,]    2    4
## 
## , , 2
## 
##      [,1] [,2]
## [1,]    5    7
## [2,]    6    8
## 
## , , 3
## 
##      [,1] [,2]
## [1,]    9   11
## [2,]   10   12
\end{verbatim}

\begin{Shaded}
\begin{Highlighting}[]
\NormalTok{arr2 }\OtherTok{\textless{}{-}} \FunctionTok{array}\NormalTok{(var1, }\AttributeTok{dim=}\FunctionTok{c}\NormalTok{(}\DecValTok{6}\NormalTok{,}\DecValTok{2}\NormalTok{))      }\CommentTok{\# 2차원 배열 생성}
\NormalTok{arr2}
\end{Highlighting}
\end{Shaded}

\begin{verbatim}
##      [,1] [,2]
## [1,]    1    7
## [2,]    2    8
## [3,]    3    9
## [4,]    4   10
## [5,]    5   11
## [6,]    6   12
\end{verbatim}

\begin{Shaded}
\begin{Highlighting}[]
\NormalTok{arr3 }\OtherTok{\textless{}{-}} \FunctionTok{array}\NormalTok{(var1, }\AttributeTok{dim=}\FunctionTok{c}\NormalTok{(}\DecValTok{2}\NormalTok{,}\DecValTok{2}\NormalTok{,}\DecValTok{3}\NormalTok{,}\DecValTok{1}\NormalTok{))  }\CommentTok{\# 4차원 배열 생성}
\NormalTok{arr3}
\end{Highlighting}
\end{Shaded}

\begin{verbatim}
## , , 1, 1
## 
##      [,1] [,2]
## [1,]    1    3
## [2,]    2    4
## 
## , , 2, 1
## 
##      [,1] [,2]
## [1,]    5    7
## [2,]    6    8
## 
## , , 3, 1
## 
##      [,1] [,2]
## [1,]    9   11
## [2,]   10   12
\end{verbatim}

\begin{Shaded}
\begin{Highlighting}[]
\NormalTok{arr4 }\OtherTok{\textless{}{-}} \FunctionTok{array}\NormalTok{(var1, }\AttributeTok{dim=}\FunctionTok{c}\NormalTok{(}\DecValTok{2}\NormalTok{,}\DecValTok{2}\NormalTok{,}\DecValTok{2}\NormalTok{,}\DecValTok{2}\NormalTok{))}
\NormalTok{arr4}
\end{Highlighting}
\end{Shaded}

\begin{verbatim}
## , , 1, 1
## 
##      [,1] [,2]
## [1,]    1    3
## [2,]    2    4
## 
## , , 2, 1
## 
##      [,1] [,2]
## [1,]    5    7
## [2,]    6    8
## 
## , , 1, 2
## 
##      [,1] [,2]
## [1,]    9   11
## [2,]   10   12
## 
## , , 2, 2
## 
##      [,1] [,2]
## [1,]    1    3
## [2,]    2    4
\end{verbatim}

\begin{center}\rule{0.5\linewidth}{0.5pt}\end{center}

\hypertarget{uxb9acuxc2a4uxd2b8}{%
\subsection{2.5 리스트}\label{uxb9acuxc2a4uxd2b8}}

리스트(List)는 다차원 데이터 저장 구조다. 배열과 다른 점은 키와 값
쌍으로 저장되며 값에 해당하는 데이터가 벡터, 행렬, 배열, 리스트 등
어떠한 데이터 구조의 데이터도 가능하다는 점이다.

\begin{Shaded}
\begin{Highlighting}[]
\NormalTok{v\_data }\OtherTok{\textless{}{-}} \FunctionTok{c}\NormalTok{(}\StringTok{"02{-}111{-}2222"}\NormalTok{, }\StringTok{"01022223333"}\NormalTok{)            }\CommentTok{\# 벡터}
\NormalTok{m\_data }\OtherTok{\textless{}{-}} \FunctionTok{matrix}\NormalTok{(}\FunctionTok{c}\NormalTok{(}\DecValTok{21}\SpecialCharTok{:}\DecValTok{26}\NormalTok{), }\AttributeTok{nrow=}\DecValTok{2}\NormalTok{)                   }\CommentTok{\# 행렬}
\NormalTok{a\_data }\OtherTok{\textless{}{-}} \FunctionTok{array}\NormalTok{(}\FunctionTok{c}\NormalTok{(}\DecValTok{31}\SpecialCharTok{:}\DecValTok{36}\NormalTok{), }\AttributeTok{dim=}\FunctionTok{c}\NormalTok{(}\DecValTok{2}\NormalTok{,}\DecValTok{2}\NormalTok{,}\DecValTok{2}\NormalTok{))              }\CommentTok{\# 배열}
\NormalTok{d\_data }\OtherTok{\textless{}{-}} \FunctionTok{data.frame}\NormalTok{(}\AttributeTok{address =} \FunctionTok{c}\NormalTok{(}\StringTok{"seoul"}\NormalTok{, }\StringTok{"busan"}\NormalTok{),  }\CommentTok{\# 데이터프레임}
                     \AttributeTok{name =} \FunctionTok{c}\NormalTok{(}\StringTok{"Lee"}\NormalTok{, }\StringTok{"Kim"}\NormalTok{), }\AttributeTok{stringsAsFactors =}\NormalTok{ F)}

\CommentTok{\# list(키1=값, 키2=값, ...,)와 같이 키와 값 쌍으로 리스트 생성}
\NormalTok{list\_data }\OtherTok{\textless{}{-}} \FunctionTok{list}\NormalTok{(}\AttributeTok{name=}\StringTok{"홍길동"}\NormalTok{,}
                  \AttributeTok{tel=}\NormalTok{v\_data,}
                  \AttributeTok{score1=}\NormalTok{m\_data,}
                  \AttributeTok{score2=}\NormalTok{a\_data,}
                  \AttributeTok{friends=}\NormalTok{d\_data)}
\NormalTok{list\_data}
\end{Highlighting}
\end{Shaded}

\begin{verbatim}
## $name
## [1] "홍길동"
## 
## $tel
## [1] "02-111-2222" "01022223333"
## 
## $score1
##      [,1] [,2] [,3]
## [1,]   21   23   25
## [2,]   22   24   26
## 
## $score2
## , , 1
## 
##      [,1] [,2]
## [1,]   31   33
## [2,]   32   34
## 
## , , 2
## 
##      [,1] [,2]
## [1,]   35   31
## [2,]   36   32
## 
## 
## $friends
##   address name
## 1   seoul  Lee
## 2   busan  Kim
\end{verbatim}

\begin{Shaded}
\begin{Highlighting}[]
\CommentTok{\# 리스트이름$키}
\NormalTok{list\_data}\SpecialCharTok{$}\NormalTok{name  }\CommentTok{\# list\_data에서 name키와 쌍을 이루는 데이터}
\end{Highlighting}
\end{Shaded}

\begin{verbatim}
## [1] "홍길동"
\end{verbatim}

\begin{Shaded}
\begin{Highlighting}[]
\NormalTok{list\_data}\SpecialCharTok{$}\NormalTok{tel   }\CommentTok{\# list\_data에서 tel키와 쌍을 이루는 데이터}
\end{Highlighting}
\end{Shaded}

\begin{verbatim}
## [1] "02-111-2222" "01022223333"
\end{verbatim}

\begin{Shaded}
\begin{Highlighting}[]
\CommentTok{\# 리스트이름[숫자]}
\NormalTok{list\_data[}\DecValTok{1}\NormalTok{]  }\CommentTok{\# list\_data에서 첫 번째 서브 리스트}
\end{Highlighting}
\end{Shaded}

\begin{verbatim}
## $name
## [1] "홍길동"
\end{verbatim}

\end{document}
